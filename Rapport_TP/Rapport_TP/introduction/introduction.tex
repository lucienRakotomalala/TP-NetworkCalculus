\chapter*{Introduction}
\addcontentsline{toc}{chapter}{Introduction}
\label{chap:Intro}

Nous allons vous présenter dans ce rapport nos travaux réalisé pour le TP du module Réseaux Temps réel : \textbf{BE : Network Calculus appliqué au réseau AFDX}. Nous avons pendant ces travaux pu étudier un réseau complexe de type AFDX avec la théorie du \emph{Network Calculus}. Cette théorie fait appel à l'algèbre (min, +) qui nous a été introduite pendant les cours du Bloc de Réactivité.

Nous avons commencé notre étude par le premier nœud du réseau, pour pouvoir appliquer la théorie sur un seul nœud. Puis, nous allons complexifier notre étude en étudiant les courbes d'arrivées et de sorties de chaque nœud ainsi que les pire temps de traversé et les pires quantités de données que chaque nœud pourra amasser.

Enfin, nous modifierons un flux du système pour essayer de voir ces conséquences sur le reste des nœuds et sur les résultats que nous aurons trouver précédemment.