\chapter{Etude du réseau complet}
Dans ce chapitre, nous allons étendre l'étude précédente à l'ensemble du réseau.
Nous allons commencer par une étude générale du réseau puis, nous étudierons les pire temps de traversé (WCTT) et les pires différences des volumes de données du réseau (backlog). Pour fini, nous verrons comment améliorer le WCTT et le backlog du réseaux.
\section{Analyse générale du réseau}
Dans un premier temps, nous allons calculer les courbes d'arrivées des flux $v_3$, $v_4$ et $v_5$ dans B et C puis les courbes de service des ports de sorties de B et C.
\subsection{Courbes d’arrivées de B et C}

\subsection{Courbes de service de B et C}

\subsection{Étude de l'ensemble du réseau}


\section{Borne sur les pires temps de traversée et sur le pire backlog}

\subsection{Pire délai de traversée de bout-en bout}

\subsection{Pire backlog du réseau}

\section{Amélioration du pire délai de traversée de bout-en bout}
\subsection{Dépendance des flux et tracé des nouvelles courbes}
\subsection{Re-calcul des délais pire cas et conclusion}