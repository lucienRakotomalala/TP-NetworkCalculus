\chapter{Etude du réseau complet}
Dans ce chapitre, nous allons étendre l'étude précédente à l'ensemble du réseau.
Nous allons commencer par une étude générale du réseau puis, nous étudierons les pire temps de traversé (WCTT) et les pires différences des volumes de données du réseau (backlog). Pour fini, nous verrons comment améliorer le WCTT et le backlog du réseaux.
\section{Analyse générale du réseau}
Dans un premier temps, nous allons calculer les courbes d'arrivées de B et C, puis les courbes de service des ports de sorties de B et C.
\subsection{Courbes d’arrivées de B et C}\label{sub:courbesArriveesB-C}
Nous avons calculé les courbes d'arrivées des flux $v_3$, $v_4$ et $v_5$, c'est-à-dire le volume de données qui arrive par intervalle de temps dans les entrées du bloc C et par la troisième entrée du bloc B. Respectivement : 
\begin{description}
\item[$\alpha_3^{C1}$ ] Qui est la même que $\alpha_1^{A1}$, dans le chapitre précédent, figure \ref{fig:CA_1_2}, page \pageref{fig:CA_1_2}. En effet $v_1$ et $v_3$ ont les mêmes caractéristiques (BAG et $s_{max}$). Elle est de type sceau percé.

\item[$\alpha_4^{C1}$ ] Qui est la même que $\alpha_2^{A1}$, dans le chapitre précédent, figure \ref{fig:CA_1_2}, page \pageref{fig:CA_1_2}. En effet $v_2$ et $v_4$ ont les mêmes caractéristiques (BAG et $s_{max}$). Elle est de type sceau percé.

\item[$\alpha_5^{B1}$ ] Qui doit réponde aux  caractéristiques suivantes : BAG$=128 ms$ et $s_{max}=800$ octets. Elle est représentée figure \ref{fig:CA_5}. Elle est de type sceau percé.
\end{description}

\begin{figure}[!ht]
\centering
\includegraphics[width = .6\textwidth]{./II/images/alpha_5.png}
\caption{\label{fig:CA_5}Courbe d'arrivée $\alpha$ du flux $v_5$ (bleu)}
\end{figure} 
\subsection{Courbes de service de B et C}\label{sub:courbesServiceB-C}
Nous allons voir ici, par la même méthode que celle utilisé dans le chapitre précédent, les courbes de services des deux sorties de B et la sortie de C. Voici, en commençant par la courbe de service du n\oe ud C :
\begin{description}
\item[$\beta^{C1}$] Qui est la même que $\beta^A$, dans le chapitre précédent, figure \ref{fig:serviceA}, page \pageref{fig:serviceA}. En effet les n\oe uds A et C ont les mêmes caractéristiques, la courbe de service est donc identique.
\item[$\beta^{B1}$ et $\beta^{B2}$] Les courbes de services de B sont identiques à celle de C car les n\oe uds ont tous les mêmes propriétés. 
\end{description}

\subsection{Étude de l'ensemble du réseau}
%1)) Calculer alpha^A1, la courbe d’arrivée du port de sortie 1 de A à partir des courbes d’arrivée par flux alpha_1^A1 1 et alpha_2^A1.
% alpha'^A déja fait avant en flux séparés 
% alpha'^C avec alpha^C = alpha_1^C + alpha_2^C et beta_C
% alpha'^B1 avec alpha^C = alpha_1^C + alpha_2^C et beta_C
Pour réaliser une étude entière du réseau, nous allons établir les courbes d'arrivées $\alpha'$ de tous les ports de sortie du réseau. Nous souhaitons dans un premier temps récapituler chaque sorties des blocs dans laquelle nous allons séparer chaque flux. Pour les sorties de $A$, nous avons déjà calculer les données maximales disponibles sur ces ports de sorties (\ref{sub:sortiesAs}) :
\begin{itemize}
\item $\alpha'^{A}_1 = \alpha^{A1}_1 \varoslash \beta$
\item $\alpha'^{A}_2 = \alpha^{A1}_2 \varoslash \beta$
\end{itemize}
Pour le nœud $C$, nous pouvons identifier les mêmes fonctions que pour $A$ avec les courbes d'arrivées et de services établis respectivement en (\ref{sub:courbesArriveesB-C}) et (\ref{sub:courbesServiceB-C}) : 
\begin{itemize}
\item $\alpha'^{C}_1 = \alpha^{C1}_1 \varoslash \beta$
\item $\alpha'^{C}_2 = \alpha^{C1}_2 \varoslash \beta$
\end{itemize}
Enfin, pour le nœud $B$, nous allons réutiliser les valeurs des ports de sortie de $A$ et $C$ ainsi que la valeur du port 3 de $B$ de  pour établir les données maximales susceptibles de parvenir sur les ports de sorties de $B$. Pour le port 1 de B : \begin{itemize}
\item $\alpha'^{B1}_1 = \alpha'^{A}_1 \varoslash \beta$
\item $\alpha'^{B1}_2 = \emptyset$
\item $\alpha'^{B1}_3 = \alpha'^{C}_1 \varoslash \beta$
\item $\alpha'^{B1}_4 = \alpha'^{C}_2 \varoslash \beta$
\item $\alpha'^{B1}_5 = \alpha'^{B1}_5 \varoslash \beta$
 \end{itemize}
 Et pour le port 2, nous obtenons : \begin{itemize}
\item $\alpha'^{B2}_1 = \alpha'^{B2}_3 = \alpha'^{B2}_4 = \alpha'^{B2}_5 = \emptyset$
\item $\alpha'^{B2}_2 = \alpha'^{A}_2 \varoslash \beta$
 \end{itemize}

Nous avons noté ici les courbes de service avec $\beta$ car nous avons déterminé qu'elles étaient toutes identiques.

Sur les figures \ref{fig:alpha1PB1}, \ref{fig:alpha3PB1}, \ref{fig:alpha4PB1} et \ref{fig:alpha5PB1} sont tracées les courbes de sortie de la première sortie de B. \\

La figure \ref{fig:alpha2PB2} contient le tracé la courbe de sortie de la seconde sortie de B.

\begin{figure}[!ht]%
% alpha1PB1 et alpha3PB1
\begin{minipage}{.48\textwidth}%
\centering%
\noindent\includegraphics[width = \textwidth]{./II/images/alpha1PB1.png}%
\caption{\label{fig:alpha1PB1}Courbe de sortie $\alpha_{1}^{'B1}$ (bleu), $\alpha_{1} ^{'A}$ (vert) et la courbe de service $\beta_B$ (rouge) de B.}%
\end{minipage}\hfill%
\begin{minipage}{.48\textwidth}%
\centering%
\noindent\includegraphics[width = \textwidth]{./II/images/alpha3PB1.png}%
\caption{\label{fig:alpha3PB1}Courbe de sortie $\alpha_3 ^{'B1}$ (bleu), $\alpha_1^{'C}$ (vert) et la courbe de service $\beta_B$ (rouge) de B.}%
\end{minipage}\vspace{5mm}\newline 
\end{figure}
\begin{figure}[!ht]
% alpha4PB1 et alpha5PB1
\begin{minipage}{.48\textwidth}%
\centering%
\noindent\includegraphics[width = \textwidth]{./II/images/alpha4PB1.png}%
\caption{\label{fig:alpha4PB1}Courbe de sortie $\alpha_4^{'B1}$ (bleu), $\alpha_2^{'C}$ (vert) et la courbe de service $\beta_B$ (rouge) de C.}%
\end{minipage}\hfill%
\begin{minipage}{.48\textwidth}%
\centering%
\noindent\includegraphics[width = \textwidth]{./II/images/alpha5PB1.png}%
\caption{\label{fig:alpha5PB1}Courbe de sortie $\alpha_5^{'B1}$ (bleu), $\alpha_5^{B}$ (vert) et la courbe de service $\beta_B$ (rouge) de C.}%
\end{minipage}%
\end{figure} 

\begin{figure}[!ht]%
%  alpha2PB2
\centering%
\noindent\includegraphics[width = .4\textwidth]{./II/images/alpha2PB2.png}%
\caption{\label{fig:alpha2PB2}Courbe de sortie $\alpha_2^{'B2}$ (bleu), $\alpha_2^{'A}$ (vert) et la courbe de service $\beta_B$ (rouge) de C.}%
\end{figure}
%2)) Calculer le délai pire-cas tau_A1 et le backlog pire-cas mu_A1 du port de sortie de A à partir de alpha^A1 et de beta^A.
\newpage 
Nous avons ensuite calculé les délais pire-cas ($\tau$) ainsi que les backlogs($\mu$) pire cas de chaque flux pour chaque nœuds qu'ils traversent. Nous obtenons les valeurs suivantes. Le calcul a été effectué à partir de la même méthode evoqué dans la partie \ref{eqn:delai-backlog}.
\begin{center}
\begin{tabular}{|c|c|c|}
\hline
Virtual Link& $\tau$(ms)& $\mu$(octets)\\
\hline
$\alpha^{C1} $   & $0.029$& $168.3$\\
$\alpha^{C2} $ 	 & $0.084$& $847.4$\\
$\alpha^{B1}_1 $ & $0.029$& $169.6$\\
$\alpha^{B1}_3 $ & $0.029$& $169.6$\\
$\alpha^{B1}_4 $ & $0.084$& $847.4$\\
$\alpha^{B1}_5 $ & $0.139$& $157.2$\\
$\alpha^{B2}_2 $ & $0.084$& $847.4$\\
\hline
\end{tabular}
\end{center}
Maintenant, nous allons effectuer la même étude mais cette fois ci en sérialisant les \emph{Virtual Links} qui peuvent être sérialisé. Nous observons que $v3$ et $v4$ sont tous les deux liés par $C$ et ne sont plus séparés ensuite. Nous pouvons donc réécrire la sortie de C comme étant :
\begin{align*}
&\alpha'^C_s = (\alpha^C_1 + \alpha^C_2) \land \lambda\\ 
&\alpha'^{B1}_{34s} = \alpha'^C_s \varoslash \beta 
\end{align*} où $\lambda$ est le débit des ports de sortie, i.e la pente de la courbe de service. Nous obtenons alors une estimation maximale des données sur le port de sortie de $C$ qui change les délais et backlog pire cas des sorties de $B$ et $C$ en :

\begin{align*}
\alpha'^C \text{ qui découle de } \alpha'^{C}_1 \text{ et } \alpha'^C_2
			& \text{: Délai pire cas   : }\tau = 	0.097 \text{ ms}\\
			& \text{  Backlog pire cas : }\mu = 1015.76\text{ octets}\\
\alpha'^B_{34} \text{ qui découle de } \alpha'^{B1}_3 \text{ et } \alpha'^{B1}_4
			& \text{: Délai pire cas   : }\tau = 	0.016 \text{ ms}\\
			& \text{  Backlog pire cas : }\mu = 200\text{ octets}\\
\end{align*}

\begin{figure}[!ht]
\centering%
\noindent\includegraphics[width = .5\textwidth]{./II/images/alpha_34s.png}%
\caption{\label{fig:alphaPB34}Courbe de sortie $\alpha_{34s}^{'B1}$ (rouge), $\alpha^{'C}_s$ (vert) et la courbe de service $\beta_B$ (vert)}
\end{figure}
%3)) Calculer alpha_1^B1 et alpha_2^B2 les courbes d’arrivée des ports de sortie 1 et 2 de B par flux.

%4)) Calculer alpha^B1 et alpha^B2 les courbes d’arrivée des ports de sortie de B en prenant en compte les flux de manière séparée.

%5)) Calculer alpha^B1 et alpha^B2 les courbes d’arrivée des ports de sortie de B en prenant en compte la sérialisation des flux.


\section{Borne sur les pires temps de traversée et sur le pire backlog}

\subsection{Pire délai de traversée de bout-en-bout}
Le pire délai de traversée de bout-en-bout est la somme pour chaque flux de tous les pires temps de traversé du port initial à la sortie.
Nous avons donc pour chaque \emph{virtual link} $v_i, i \in \left\lbrace 1,2,3,4,5\right\rbrace$, calculé le pire temps de traversée $WCTT_i$ :
\begin{equation}
\begin{array}{lclcll}
WCTT_1	&=&	\tau_{A1} + \tau_{B11} 	&=& 58,8269&\mu s\\
WCTT_2	&=&	\tau_{A2} + \tau_{B22} 	&=& 167,5539&\mu s\\
WCTT_3	&=&	\tau_{C1} + \tau_{B31} 	&=& 58,8269	&\mu s\\
WCTT_4	&=&	\tau_{C2} + \tau_{B41} 	&=& 167,554	&\mu s\\
WCTT_5	&=&	\tau_{B51} 				&=& 139,76	&\mu s\\
\end{array}
\end{equation}
Tous les temps de traversée sont supérieur à deux fois la latence technologique, qui, pour $v_1, v_2, v_3$ et $v_4$ est le temps minimal de transit technologique.\\

Le pire temps de traversé est donc le maximum de ces valeurs et vaut $167,554 \text{ }\mu s$ et il correspond au temps de traversé d'un octet sur le \emph{virtual link} $v4$.
\subsection{Pire backlog du réseau}
Le pire backlog du réseau est le noeud pour lequel la somme des backlog est la plus grande.
Nous avons calculer le backlog de chaque signaux et avons trouvé que c'était le noeud B qui présente le pire backlog : 
\begin{equation}
\mu_B = \mu_{B11} + \mu_{B31} + \mu_{B41} + \mu_{B51} + \mu_{B22}  = 3582,2323 \text{ octets}
\end{equation}
\section{Amélioration du pire délai de traversée de bout-en bout}
\subsection{Dépendance des flux et tracé des nouvelles courbes}
Les flux sortant de C, $\alpha_1^{'C}$ et $\alpha_2^{'C}$ sortent tous deux dans la sortie 1 de B $\alpha_1^{'B}$. Ils peuvent donc être traité comme un seul flux et être sérialisé.
Les courbes de services des n\oe uds ne varient pas. Les courbes d'arrivées et de sorties de A non plus.
Voici les nouvelles courbes pour B et C :
\begin{figure}[!ht]%
%  alpha2PB2
\centering%
\noindent\includegraphics[width = .6\textwidth]{./II/images/alphaSerialise.png}%
\caption{\label{fig:alphaSerialise}Courbe de sortie $\alpha_3,4^{'B1}$ (rouge), $\alpha^{'C}$ (bleu) et la courbe de service $\beta_B$ (vert) de C.}%
\end{figure}

\subsection{Re-calcul des délais pire cas et conclusion}
Ce changement d'approche a modifié certains temps de transferts de bout-en-bout. 
\begin{equation}
\begin{array}{lclcll}
WCTT_1	&=&	\tau_{A1} + \tau_{B11} 	&=& 58,8269&\mu s\\
WCTT_2	&=&	\tau_{A2} + \tau_{B22} 	&=& 167,5539&\mu s\\
WCTT_3	&=&	\tau_{C} + \tau_{B(3,4)1} 	&=& 113,12&\mu s\\
WCTT_4	&=&	\tau_{C} + \tau_{B(3,4)1} &=& 113,12	&\mu s\\
WCTT_5	&=&	\tau_{B51} 				&=& 139,76	&\mu s\\
\end{array}
\end{equation}
Le pire backlog est toujours sur le n\oe ud B, avec une valeur $2595,0403$ octets.
On remarque que la sérialisation permet un gain de 38\% sur le backlog.

% conclusion